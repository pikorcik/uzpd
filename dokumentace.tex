\documentclass[a4paper, 12pt]{article}
\usepackage[total={17cm,25cm}, top=2.5cm, left=2.5cm, right=2.5cm,  includefoot]{geometry}
\usepackage[utf8]{inputenc}
\usepackage{array}
\usepackage{multirow}
\usepackage{hhline}
\usepackage{gensymb}
\usepackage{graphicx}
\graphicspath{ {} }
\usepackage[czech]{babel}
\usepackage{enumitem}
\usepackage{pdfpages}
\usepackage{amsmath}
\usepackage{verbatim}
\usepackage{listings}
\usepackage{hyperref}
\usepackage{amssymb}
\usepackage{color}
\usepackage{upquote}

\def\UrlBreaks{\do\/\do-\do.\do=\do_\do?\do\&\do\%}


\definecolor{mygreen}{rgb}{0,0.6,0}
\definecolor{mygray}{rgb}{0.5,0.5,0.5}
\definecolor{mybackground}{rgb}{0.92,0.92,1}

\lstset{
  backgroundcolor=\color{mybackground},
  commentstyle=\color{mygreen},
  firstnumber=1,
  basicstyle=\footnotesize,  
  numbers=left,
  numbersep=5pt,
  numberstyle=\tiny\color{mygray},
  keepspaces=true, 
  breakatwhitespace=false, 
  breaklines=true, 
  language=sql,
  extendedchars=false,
  escapeinside=<>
}



\pagestyle{empty} % vypne číslování stránek




\usepackage[OT2,OT1]{fontenc}
\newcommand\cyr
{
\renewcommand\rmdefault{wncyr}
\renewcommand\sfdefault{wncyss}
\renewcommand\encodingdefault{OT2}
\normalfont
\selectfont
}
\DeclareTextFontCommand{\textcyr}{\cyr}
\def\cprime{\char"7E }
\def\cdprime{\char"7F }
\def\eoborotnoye{\char’013}
\def\Eoborotnoye{\char’003}
\setlength{\parindent}{1em} 
%\setlength{\parskip}{0.5ex}


\begin{document}

\begin{titlepage}
\begin{center}
\Huge
\vspace*{5.5cm}
Úvod do zpracování prostorových dat\\
\vspace{0.2cm}

\Large  
Kvalita bydlení ve vybraných částech Prahy\\
\vspace{0.2cm}

\normalsize  
Zimní semestr 2018/2019\\
\vspace{13cm}
\end{center}

\begin{flushright}
\Large
Tereza Kulovaná \\
Markéta Pecenová \\
\end{flushright}

\end{titlepage}


\pagestyle{plain}     % zapne obyčejné číslování
\setcounter{page}{1}  % nastaví čítač stránek znovu od jedné

\tableofcontents
\newpage

\section{Zadání}
\subsection{Zadání projektu}
Navrhněte a vytvořte tématické vrstvy (např. vodní toky, vodní plochy, lesy, silnice, železnice apod.) na základě dat \textit{OpenStreetMap} a další otevřených zdrojů. Aplikujte testy datové integrity a odstraňte případné nekonzistence v datech. Vytvořte tutoriál - tj. sadu atributových a prostorových dotazů nad databází \textbf{pgis\_uzpd}.

\subsection{Zvolené téma}
Jako téma pro semestrální projekt byla zvolena analýza kvality bydlení v Praze. Jelikož je toto téma velmi obecné a obsáhlé, bylo nutné zvolit užší zaměření. Z mnoha podkladů, které byly k dispozici, byly vybrány datové vrstvy, které nějak souvisely s občanskou vybaveností nebo zdravím obyvatel Prahy. Výsledným produktem je sada atributových a prostorových dotazů, které vyhodnocují kvalitu místa pro život na základě zvolených vstupních ukazatelů kvality. 
\clearpage

\section{Data}
V rámci projektu bylo nutné získat potřebná vstupní data, nahrát je do databáze a zajistit jejich konzistenci. Nad těmito daty pak byly následně provedeny prostorové dotazy.

\subsection{Data a tématické vrstvy}
Data použitá v rámci projektu pochází ze tří zdrojů: ze schématu \textit{ruian\_praha}, portálů \textit{Otevřená data} a \textit{Geoportál Praha}.

\subsubsection{ruain\_praha}
Schéma \textit{ruian\_praha} je součástí databáze \textbf{pgis\_uzpd} a je v souřadnicovém systému JTSK.\\

Vrstvy:
\begin{itemize}
\item \textsl{adresni\_mista} (adresnimista)
\item \textsl{obvody} (spravniobvody)
\end{itemize}

\subsubsection{Otevřená data}
Veškeré datové vrstvy, které byly staženy z portálu \textit{Otevřená data}, mají uvedené jako  pos\-ky\-tovatele \textsl{HLAVNÍ MĚSTO PRAHA} a jsou vztažena pouze na území Prahy. Geometrie všech vrstev je reprezentována bodem. Data byla stažena ve formátu geoJSON a až na poslední uvedenou vrstvu byla v souřadnicovém systému WGS84 (S-JTSK nebyl k dispozici). Pouze \textsl{Vstupy do metra} byly staženy přímo v S-JTSK.\\

Vrstvy:  
\begin{itemize}
\item \textsl{detska\_hriste} (Dětská hřiště Praha, [\href{https://data.gov.cz/datov%C3%A1-sada?iri=https%3A%2F%2Fdata.gov.cz%2Fzdroj%2Fdatov%C3%A1-sada%2Fhttp---opendata.praha.eu-api-3-action-package_show-id-detska-hriste-praha}{Zdroj}])
\item \textsl{zdrav\_zarizeni} (Lékárny a zdravotnická zařízení v Praze, [\href{https://data.gov.cz/datov%C3%A1-sada?iri=https%3A%2F%2Fdata.gov.cz%2Fzdroj%2Fdatov%C3%A1-sada%2Fhttp---opendata.praha.eu-api-3-action-package_show-id-lekarny-a-zdravotnicka-zarizeni-v-praze}{Zdroj}])
\item \textsl{metro} (Vstupy do metra, [\href{https://data.gov.cz/datov%C3%A1-sada?iri=https%3A%2F%2Fdata.gov.cz%2Fzdroj%2Fdatov%C3%A1-sada%2Fhttp---opendata.praha.eu-api-3-action-package_show-id-ipr-prazska_integrovana_doprava_-_vstupy_do_metra}{Zdroj}])
\end{itemize}

\subsubsection{Geoportál Praha}
Ze stránek \textit{Geoportál Praha} byla stažena tématická data zachycující rozmístění košů na tříděný odpad a data zaplaveného území Prahy při povodních v roce 2013. Data byla stažena ve formátu geoJSON v systému JTSK. Geometrie rozmíštění košů je \textit{bod} a zaplaveného území \textit{polygon}. Pro povodňová data bylo k dispozici ke stažení vrstev mnohem více, například záplavové čáry pro stoletou/padesátiletou/dvacetiletou vodu atd. Pro zjednodušení byla stažena data, který byla ucelená (zachycující jedno období) a nejaktuálnější.\\

Vrstvy:
\begin{itemize}
\item \textsl{odpad} (Mapa košů na tříděný odpad, [\href{http://www.geoportalpraha.cz/cs/opendata/DBD0F925-78BF-478D-8FB8-8B3BDE9BD581#.XFDFzGlCfIV}{Zdroj}])
\item \textsl{zaplavy2013} (Záplavové území 2013, [\href{http://www.geoportalpraha.cz/cs/opendata/C121457E-5450-42D0-9009-204D8D899A06#.XE9PwWlCfIV}{Zdroj}])
\end{itemize}


\section{Zpracování dat}
\subsection{ruian\_praha}
\subsubsection*{Adresní místa}
Data byla stažena ze schématu \textit{ruian\_praha} (vrstva \textit{adresnimista}) do nově vytvořené tabulky \textsl{adresni\_mista}. Z původních dat byly zkopírovaný pouze sloupečky \textsl{kód}, \textsl{číslo popisné}, \textsl{číslo orientační} a sloupeček s geometrií \textsl{geom}. 
\begin{lstlisting}
CREATE TABLE adresni_mista AS
    SELECT kod, cislodomovni, cisloorientacni, geom
    FROM ruian_praha.adresnimista
\end{lstlisting}

Následně byly nad sloupečkem \textsl{kód} nastaven primární klíč a nad sloupečkem \textsl{geom} prostorový index.
\begin{lstlisting}
ALTER TABLE adresni_mista ADD PRIMARY KEY(kod)

CREATE index adresy_index ON adresni_mista(geom)
\end{lstlisting}

Na závěr byla provedena kontrola validity geometrie, která vyšla negativní.
\begin{lstlisting}
SELECT kod FROM adresni_mista WHERE NOT st_isvalid(geom)
\end{lstlisting}
Při zběžném prohlédnutí již zpracovaných dat bylo zjištěno, že výše uvedená funkce na kontrolu validity nefunguje zcela správně, jelikož neodstranila záznamy které měly ve sloupečku pro geometrii uvedenou hodnotu \textit{NULL}. Bylo tedy nutné provést dodatečné pročištění dat níže uvedeným příkazem:
\begin{lstlisting}
DELETE FROM adresni_mista WHERE geom IS NULL
\end{lstlisting}

\subsubsection*{Správní obvody Prahy}
Data byla stažena ze schématu \textit{ruian\_praha} (vrstva \textit{spravniobvody}) do nově vytvořené tabulky \textsl{obvody}. Geometrie dat je \textit{multipolygon} a představují jednotlivé pražské správní obvody. Z původních dat byly zkopírovaný pouze sloupečky \textsl{ogc\_fid}, \textsl{název} a sloupeček s geometrií \textsl{geom}. 
\begin{lstlisting}
CREATE TABLE obvody AS
    SELECT ogc_fid, nazev, geom
    FROM ruian_praha.spravniobvody
\end{lstlisting}

Následně byly nad sloupečkem \textsl{ogc\_fid} nastaven primární klíč.
\begin{lstlisting}
ALTER TABLE obvody ADD PRIMARY KEY(ogc_fid)
\end{lstlisting}

Na závěr byla provedena kontrola validity geometrie, která vyšla negativní.
\begin{lstlisting}
SELECT nazev FROM obvody WHERE NOT st_isvalid(geom)
\end{lstlisting}

\subsection{Otevřená data}
Všechna data z tohoto zdroje byla stažena ve formátu geoJSON a byla převážně v systému WGS84 (výjimka: vrstva \textit{metro} bylo od počátku v S-JTSK). Postup při zpracování těchto dat byl obdobný. Data byla importována do příslušného schématu pomocí nástroje \textit{ogr2ogr} a přímo v databázi byly v tabulkách promazány nepotřebné sloupečky a přejmenován sloupeček s geometrií, který po nahrání dávkou získal uživatelsky nepřívětivý název \textit{wkb\_ge\-o\-metry}.\\

Během importu do databáze bylo u některých dat nutné provést transformaci ze systému WGS84 (EPSG: 4326) do S-JTSK (EPSG: 5514). Transformaci zajišťovala část kódu \linebreak \textsl{-t\_srs 'EPSG:5514'}. Nebyla-li transformace nutná, nahradila se tato část kódu za \textsl{-a\_srs 'EPSG:5514'}. Při importu byl automaticky vytvořen sloupeček s primárním klíčem \textsl{ogc\_fid} a byly vytvořeny prostorové indexy. Níže je uveden obecný kód pro nahrání souboru do příslušného schématu databáze (s transformací do S-JTSK):

\begin{lstlisting}
ogr2ogr -f "PostgreSQL" PG:"dbname=pgis_uzpd user=uzpd18_d host=geo102.fsv.cvut.cz" -t_srs 'EPSG:5514' "input.json" -nln uzpd18_d.table
\end{lstlisting}
Jelikož geometrie všech vrstev byla \textit{bod}, pro kontrolu validity byla pro všechny vrstvy použita stejná funkce:
\begin{lstlisting}
SELECT ogc_fic FROM nazev_tabulky WHERE NOT st_isvalid(geom)
\end{lstlisting}

\subsubsection*{Dětská hřiště Praha}
Příkaz pro smazání sloupečků, které pro další práci s daty nebyly nutné:
\begin{lstlisting}
ALTER TABLE detska_hriste
    DROP COLUMN url,
    DROP COLUMN name,
    DROP COLUMN perex,
    DROP COLUMN content,
    DROP COLUMN address,
    DROP COLUMN properties,
    DROP COLUMN image
\end{lstlisting}
Přejmenování sloupečku s geometrií:
\begin{lstlisting}
ALTER TABLE detska_hriste
    RENAME wkb_geometry TO geom
\end{lstlisting} 
Výsledná tabulka má tyto sloupce:
\begin{table}[h!]
\centering
\begin{tabular}{|c|c|c|c|}
\hline
\multicolumn{4}{|c|}{\textbf{detska\_hriste}} \\ \hline
ogc\_fid     & geom     & id     & district    \\ \hline
\end{tabular}
\end{table}

\subsubsection*{Lékárny a zdravotnická zařízení v Praze}
Příkaz pro smazání sloupečků, které pro další práci s daty nebyly nutné:
\begin{lstlisting}
ALTER TABLE zdrav_zarizeni
    DROP COLUMN id,
    DROP COLUMN address,
    DROP COLUMN email,
    DROP COLUMN web,
    DROP COLUMN telephone,
    DROP COLUMN opening_hours
\end{lstlisting}
Přejmenování sloupečku s geometrií:
\begin{lstlisting}
ALTER TABLE zdrav_zarizeni
    RENAME wkb_geometry TO geom
\end{lstlisting} 
Výsledná tabulka má tyto sloupce:
\begin{table}[h!]
\centering
\begin{tabular}{|c|c|c|c|c|}
\hline
\multicolumn{5}{|c|}{\textbf{zdrav\_zarizeni}} \\ \hline
ogc\_fid   & geom  & name  & type  & district  \\ \hline
\end{tabular}
\end{table}

\subsubsection*{Vstupy do metra}
Jelikož stažená data byla již v S-JTSK, část kódu zajišťující transformaci při importu byla nahrazena za \textsl{-a\_srs 'EPSG:5514'}. Příkaz pro smazání sloupečků, které pro další práci s daty nebyly nutné:
\begin{lstlisting}
ALTER TABLE metro
    DROP COLUMN objectid,
    DROP COLUMN vstupy_kod,
    DROP COLUMN vstupy_vest_kod,
    DROP COLUMN vstupy_vest_nazev,
    DROP COLUMN vstupy_vazba_bus,
    DROP COLUMN vstupy_vazba_csad,
    DROP COLUMN vstupy_vazba_kr,
    DROP COLUMN vstupy_vazba_pr,
    DROP COLUMN vstupy_vazba_privoz,
    DROP COLUMN vstupy_vazba_taxi,
    DROP COLUMN vstupy_vazba_tram,
    DROP COLUMN vstupy_vazba_vlak,
    DROP COLUMN zast_uzel_cislo,
    DROP COLUMN vstupy_popis,
    DROP COLUMN poskyt,
    DROP COLUMN vstupy_mimo_provoz
\end{lstlisting}
Přejmenování sloupečku s geometrií a názvu stanic:
\begin{lstlisting}
ALTER TABLE metro
    RENAME wkb_geometry TO geom,
    RENAME vstupy_uzel_nazev TO nazev,
    RENAME vstupy_linka TO linka
\end{lstlisting} 
Výsledná tabulka má tyto sloupce:
\begin{table}[h!]
\centering
\begin{tabular}{|c|c|c|c|}
\hline
\multicolumn{4}{|c|}{\textbf{metro}} \\ \hline
ogc\_fid      & geom     & nazev     & linka     \\ \hline
\end{tabular}
\end{table}


\subsection{Geoportál Praha}
Data získaná z tohoto portálu byla stažena ve formátu geoJSON v systému JTSK. Zpracování dat probíhalo obdobně jako u předchozího zdroje dat s výjimkou, že při importu nebylo nutné provádět transformaci. Obecný kód pro nahrání souboru do příslušného schématu databáze (bez transformace do S-JTSK): 
\begin{lstlisting}
ogr2ogr -f "PostgreSQL" PG:"dbname=pgis_uzpd user=uzpd18_d host=geo102.fsv.cvut.cz" -a_srs 'EPSG:5514' "input.json" -nln uzpd18_d.table
\end{lstlisting}

\subsubsection*{Mapa košů na tříděný odpad}
Příkaz pro smazání sloupečků, které pro další práci s daty nebyly nutné:
\begin{lstlisting}
ALTER TABLE odpad
    DROP COLUMN objectid,
    DROP COLUMN id,
    DROP COLUMN stationnumber,
    DROP COLUMN stationname,
    DROP COLUMN citydistrictruiancode
\end{lstlisting}
Přejmenování sloupečku s geometrií:
\begin{lstlisting}
ALTER TABLE odpad
    RENAME wkb_geometry TO geom
\end{lstlisting} 
Na závěr byla provedena kontrola validity geometrie. 
\begin{lstlisting}
SELECT ogc_fic FROM nazev_tabulky WHERE NOT st_isvalid(geom)
\end{lstlisting}
Výsledná tabulka má tyto sloupce:

\begin{table}[h!]
\centering
\begin{tabular}{|c|c|c|c|}
\hline
\multicolumn{4}{|c|}{\textbf{odpad}}     \\ \hline
ogc\_fid & geom & citydistrict & pristup \\ \hline
\end{tabular}
\end{table}

\subsubsection*{Záplavové území 2013}
Příkaz pro smazání sloupečků, které pro další práci s daty nebyly nutné:
\begin{lstlisting}
ALTER TABLE zaplava2013
    DROP COLUMN objectid,
    DROP COLUMN nazev,
    DROP COLUMN typ
\end{lstlisting}
Přejmenování sloupečku s geometrií:
\begin{lstlisting}
ALTER TABLE zaplava2013
    RENAME wkb_geometry TO geom
\end{lstlisting} 
Opět bylo nutné zkontrolovat validitu geometrie (polygonů). Níže uvedeným příkazem byly zjištěny chyby v datech:
\begin{lstlisting}
SELECT id, geom, st_isvalidreason(geom) FROM uzpd18_d.zaplava2013 WHERE NOT st_isvalid(geom)
\end{lstlisting}
Ukázalo se, že nevalidní byly 4 polygony z celkových 86. Jako příčina těchto chyb se ukázala tzv. \textit{Ring-self Intersection} polygonů. Nevalidní polygony byly opraveny vytvoře\-ním \textit{bufferu} o velikosti 0. 
\begin{lstlisting}
UPDATE uzpd18_d.zaplava2013 SET geom = st_buffer(geom, 0.0) WHERE NOT st_isvalid(geom)
\end{lstlisting}
Výsledná tabulka má tyto sloupce:

\begin{table}[h!]
\centering
\begin{tabular}{|c|c|c|c|}
\hline
\multicolumn{4}{|c|}{\textbf{zaplava2013}}    \\ \hline
ogc\_fid & geom & shape\_length & shape\_area \\ \hline
\end{tabular}
\end{table}
\clearpage

\section{SQL dotazy}
\subsubsection*{1) Kolik adresních míst bylo zatopeno během povodní v roce 2013?}
\begin{lstlisting}[language=html]
SELECT COUNT(a.kod) 
FROM uzpd_d.adresni_mista AS a
JOIN (SELECT * FROM uzpd_d.zaplava2013) AS z
ON st_intersects(z.geom, a.geom);
\end{lstlisting}
1042
\vspace{0.8cm}

\subsubsection*{2) Jaká zdravotnická zařízení byla zasažena povodněmi v roce 2013? Vypište jejich název a správní obvod, ve kterém se nacházejí.}
\begin{lstlisting}[language=html]
SELECT zz.name, zz.district 
FROM uzpd_d.zdrav_zarizeni AS zz
JOIN uzpd_d.zaplava2013 AS za 
ON st_intersects(zz.geom,za.geom);
\end{lstlisting}
Dr.Max LÉKÁRNA, praha-8
\vspace{0.8cm}

\subsubsection*{3) Kolik adresních míst leží ve vzdálenosti do 500 m od výlezu ze stanice metra Depo Hostivař?}
\begin{lstlisting}[language=html]
SELECT count(distinct a.kod) 
FROM uzpd18_d.adresni_mista AS a
JOIN uzpd18_d.metro AS m
ON st_dwithin(a.geom, m.geom, 500)
WHERE m.nazev = 'Depo Hostiva<ř>';
\end{lstlisting}
48
\vspace{0.8cm}

\subsubsection*{4) Kolik košů na tříděný odpad se nachází na území Prahy 1?}
\begin{lstlisting}[language=html]
SELECT COUNT(o.ogc_fid) 
FROM uzpd18_d.odpad AS o
JOIN (SELECT * FROM uzpd18_d.obvody WHERE nazev = 'Praha 1') AS p
ON st_intersects(p.geom, o.geom);
\end{lstlisting}
821
\vspace{0.8cm}

\subsubsection*{5) Jaké procento území Prahy zaujímá záplavové území z roku 2013?}
\begin{lstlisting}[language=html]
SELECT ROUND(
(
SELECT sum(st_area(geom)) FROM uzpd18_d.zaplava2013
)::numeric / (
SELECT sum(st_area(geom)) FROM uzpd18_d.obvody
)::numeric, 2)*100 AS procento_zaplava;
\end{lstlisting}
5\%
\vspace{0.8cm}

\subsubsection*{6) Jaká stanice metra na Praze 4 má nejvíce vstupů?}
\begin{lstlisting}[language=html]
SELECT m.nazev FROM uzpd18_d.metro AS m
JOIN (SELECT * FROM uzpd18_d.obvody 
WHERE nazev = 'Praha 4') AS p
ON st_intersects(p.geom, m.geom)
GROUP BY m.nazev
ORDER BY COUNT(m.ogc_fid) DESC
LIMIT 1;
\end{lstlisting}
Budějovická
\vspace{0.8cm}

\subsubsection*{7) Která zdravotnická zařízení leží do 10 m od stanic metra A? V které městské části se tato zařízení nacházejí?}
\begin{lstlisting}[language=html]
SELECT z.name, z.district 
FROM uzpd18_d.zdrav_zarizeni AS z
JOIN uzpd18_d.metro AS m
ON st_dwithin(z.geom, m.geom, 10)
WHERE m.linka LIKE '%A%';
\end{lstlisting}
\begin{flushleft}
Dr.Max LÉKÁRNA, praha-1\linebreak
BENU Lékárna OC Atrium Flora, praha-3
\end{flushleft}
\vspace{0.009cm}

\subsubsection*{8) Kolik dětských hřišť se nachází ve vzdálenosti do 2 kilometrů od nejbližší nemocnice?}
\begin{lstlisting}[language=html]
WITH 
    n AS (
        SELECT * FROM uzpd18_d.zdrav_zarizeni AS z 
        WHERE type LIKE '%nemocnice%')
SELECT COUNT(DISTINCT(h.ogc_fid)) 
FROM uzpd18_d.detska_hriste AS h, n
WHERE n.geom && st_expand(h.geom, 2000);
\end{lstlisting}
61
\vspace{0.8cm}

\subsubsection*{9) Kolik adresních míst má lékárnu a dětské hřiště do vzdálenosti 500 m?}
\begin{lstlisting}[language=html]
WITH
    buff_h AS (
        SELECT st_buffer(geom, 500) AS geom 
        FROM uzpd18_d.detska_hriste AS h),
    buff_z AS (
        SELECT st_buffer(geom, 500) AS geom 
        FROM uzpd18_d.zdrav_zarizeni AS z
        WHERE type = 'L<é>k<á>rna') 
SELECT COUNT(DISTINCT(kod)), a.geom FROM uzpd18_d.adresni_mista AS a
INNER JOIN buff_h ON st_intersects(a.geom, buff_h.geom)
INNER JOIN buff_z ON st_intersects(a.geom, buff_z.geom);
\end{lstlisting}
25130
\vspace{0.8cm}

\subsubsection*{10) Který pražský správní obvod má nejlepší poměr počtu košů na tříděný odpad vzhledem ke své rozloze? Uveďte název obvodu, jeho rozlohu, počet košů a poměr těchto dvou hodnot.}
\begin{lstlisting}[language=html]
WITH 
    oo AS (
        SELECT ob.nazev, count(od.ogc_fid), ob.geom FROM obvody AS ob
        JOIN uzpd18_d.odpad AS od 
        ON st_intersects(ob.geom, od.geom)
        GROUP BY ob.nazev, ob.geom
        ORDER BY COUNT(od.ogc_fid) DESC),
    area AS (
        SELECT obvody.nazev, st_area(obvody.geom), geom from obvody)
SELECT oo.nazev, ROUND(area.st_area), count, ROUND(area.st_area/count) AS ratio 
FROM oo
JOIN area ON oo.nazev = area.nazev
GROUP BY oo.nazev, area.st_area, count, area.st_area/count
ORDER BY area.st_area/count DESC
LIMIT 1;
\end{lstlisting}
Praha 22, 33660132, 70, 480859
\vspace{0.8cm}

\subsubsection*{11) Na území kterého pražského správního obvodu se nachází největší množství dětských hřišť a kolik to je?!!!!!!!!}
\begin{lstlisting}[language=html]
SELECT o.nazev, COUNT(distinct dh.ogc_fid) 
FROM uzpd18_d.obvody AS o 
JOIN uzpd18_d.detska_hriste AS dh 
ON st_intersects(dh.geom, o.geom) 
HAVING COUNT(distinct dh.ogc_fid) = (
    SELECT COUNT(distinct dh.ogc_fid) FROM uzpd18_d.obvody AS o 
    JOIN uzpd18_d.detska_hriste AS dh 
    ON st_intersects(dh.geom, o.geom) 
    GROUP BY o.nazev 
    ORDER BY COUNT(distinct dh.ogc_fid) 
    DESC LIMIT 1);
\end{lstlisting}
\begin{flushleft}
praha-3, 14 \linebreak
praha-4, 14
\end{flushleft}

\subsubsection*{12) Jaké je id nejbližšího dětského hřiště pro adresní místo s kódem 22560840? V jaké leží vzdálenosti?}
\begin{lstlisting}[language=html]
SELECT dh.id, ROUND(st_distance(a.geom, dh.geom)) AS distance 
FROM uzpd18_d.detska_hriste AS dh, uzpd18_d.adresni_mista AS a
WHERE a.kod = '22560840' AND st_distance(a.geom, dh.geom) = (
    SELECT st_distance(a.geom, dh. geom) 
    FROM detska_hriste AS dh, uzpd18_d.adresni_mista AS a
    WHERE a.kod = '22560840' 
    ORDER BY st_distance(a.geom, dh.geom)
    LIMIT 1);
\end{lstlisting}
120, 506 m
\vspace{0.8cm}

\clearpage

\section{Závěr}


\clearpage

\section{Přílohy} 

\begin{itemize}
\item Příloha č. 1: Prezentace (prezentace.pdf)
\item Příloha č. 2: SQL dávka (davka.sql)
\end{itemize}

\clearpage

\section{Zdroje}
\begin{enumerate}
\item  \textsl{Otevřená data} [online] [cit. 28. 1. 2019].\\
Dostupné z: \href{http://www.geoportalpraha.cz/cs/opendata?fbclid=IwAR3dvAz20d2Anu-nuD9A7wC3byHUKTzDGTnlgQrmi0tC-t-SbSqN7Q5x-sA#.XE8ymWlCfIX}{http://www.geoportalpraha.cz/}

\item  \textsl{Datové sady – Národní katalog otevřených dat (NKOD)} [online] [cit. 28. 1. 2019].\\
Dostupné z: \href{https://data.gov.cz/datov\%C3\%A9-sady?dotaz=&fbclid=IwAR2jg2NvnsjO7BuHAgbXSJwh6VXFbsuXs0FONcle5ZfdPpO86Z_2C3YUQsU}{https://data.gov.cz/}

\item  \textsl{Školení postGIS pro začátečníky} [online] [cit. 30. 1. 2019].\\
Dostupné z: \href{http://training.gismentors.eu/postgis-zacatecnik/}{http://training.gismentors.eu/}

\item  \textsl{Školení postGIS pro pokročilé} [online] [cit. 30. 1. 2019].\\
Dostupné z: \href{http://training.gismentors.eu/postgis-pokrocily/}{http://training.gismentors.eu/}

\item  \textsl{155UZPD / Semestrální projekt} [online] [cit. 30. 1. 2019].\\
Dostupné z: \href{http://geo.fsv.cvut.cz/gwiki/155UZPD_/_Semestr%C3%A1ln%C3%AD_projekt}{http://geo.fsv.cvut.cz/}

\item  \textsl{LaTeX/Source Code Listing} [online] [cit. 28. 1. 2019].\\
Dostupné z: \href{https://en.wikibooks.org/wiki/LaTeX/Source_Code_Listings}{https://en.wikibooks.org/}

\end{enumerate}

\end{document}
