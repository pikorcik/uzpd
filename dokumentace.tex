\documentclass[a4paper, 12pt]{article}
\usepackage[total={17cm,25cm}, top=2.5cm, left=2.5cm, right=2.5cm,  includefoot]{geometry}
\usepackage[utf8]{inputenc}
\usepackage{array}
\usepackage{multirow}
\usepackage{hhline}
\usepackage{gensymb}
\usepackage{graphicx}
\graphicspath{ {} }
\usepackage[czech]{babel}
\usepackage{enumitem}
\usepackage{pdfpages}
\usepackage{amsmath}
\usepackage{verbatim}
\usepackage{listings}
\usepackage{hyperref}
\usepackage{amssymb}
\usepackage{color}


\definecolor{mygreen}{rgb}{0,0.6,0}
\definecolor{mygray}{rgb}{0.5,0.5,0.5}
\definecolor{mybackground}{rgb}{0.92,0.92,1}

\lstset{ 
  backgroundcolor=\color{mybackground},
  commentstyle=\color{mygreen},
  firstnumber=1,
  basicstyle=\footnotesize,  
  numbers=left,
  numbersep=5pt,
  numberstyle=\tiny\color{mygray},
  keepspaces=true, 
  breakatwhitespace=false, 
  language=sql,
}


\pagestyle{empty} % vypne číslování stránek




\usepackage[OT2,OT1]{fontenc}
\newcommand\cyr
{
\renewcommand\rmdefault{wncyr}
\renewcommand\sfdefault{wncyss}
\renewcommand\encodingdefault{OT2}
\normalfont
\selectfont
}
\DeclareTextFontCommand{\textcyr}{\cyr}
\def\cprime{\char"7E }
\def\cdprime{\char"7F }
\def\eoborotnoye{\char’013}
\def\Eoborotnoye{\char’003}
\setlength{\parindent}{1em} 
%\setlength{\parskip}{0.5ex}


\begin{document}

\begin{titlepage}
\begin{center}
\Huge
\vspace*{4.5cm}
Úvod do zpracování prostorových dat\\
\vspace{0.2cm}

\Large  
Kvalita bydlení ve vybraných částech Prahy\\
\vspace{0.2cm}

\normalsize  
Zimní semestr 2018/2019\\
\vspace{14cm}
\end{center}

\begin{flushright}
\Large
Tereza Kulovaná \\
Markéta Pecenová \\
\end{flushright}

\end{titlepage}


\pagestyle{plain}     % zapne obyčejné číslování
\setcounter{page}{1}  % nastaví čítač stránek znovu od jedné

\tableofcontents
\newpage

\section{Zadání}
\subsection{Zadání projektu}
Navrhněte a vytvořte tématické vrstvy (např. vodní toky, vodní plochy, lesy, silnice, železnice apod.) na základě dat \textit{OpenStreetMap} a další otevřených zdrojů. Aplikujte testy datové integrity a odstraňte případné nekonzistence v datech. Vytvořte tutoriál - tj. sadu atributových a prostorových dotazů nad databází pgis\_uzpd.

\subsection{Zvolené téma}


\clearpage

\section{Zpracování dat}
\subsection{Data}


\subsection{Příprava databáze}

\begin{lstlisting}
CREATE TABLE adresni_mista AS
	SELECT kod, cislodomovni, cisloorientacni, geom
	FROM ruian_praha.adresnimista
\end{lstlisting}

\begin{lstlisting}
ALTER TABLE adresni_mista ADD PRIMARY KEY(kod)
\end{lstlisting}

\begin{lstlisting}
CREATE index adresy_index ON adresni_mista(geom)
\end{lstlisting}

\begin{lstlisting}
SELECT kod FROM adresni_mista WHERE not st_isvalid(geom)
\end{lstlisting}

\begin{lstlisting}
DELETE FROM adresni_mista WHERE geom IS NULL
\end{lstlisting}




\clearpage
\begin{comment}
\section{Aplikace}
V následují kapitole je představen vizuální vzhled vytvořené aplikace tak, jak ji vidí prostý uživatel.

\begin{figure}[h!]
	\centering
	\includegraphics[width=14cm]{./pictures/app_default.png}
	\caption{Výchozí vzhled aplikace po spuštění}
\end{figure}

\begin{figure}[h!]
	\centering
	\includegraphics[width=14cm]{./pictures/app_load_points.png}
	\caption{Aplikace po nahrání vstupních dat}
\end{figure}

\begin{figure}[h!]
	\centering
	\includegraphics[width=15cm]{./pictures/app_delaunay.png}
	\caption{Trojúhelníková síť}
\end{figure}

\begin{figure}[h!]
	\centering
	\includegraphics[width=15cm]{./pictures/app_contours.png}
	\caption{Vykreslení vrstevnic}
\end{figure}

\begin{figure}[h!]
	\centering
	\includegraphics[width=15cm]{./pictures/app_slope.png}
	\caption{Sklon trojúhelníků}
\end{figure}

\begin{figure}[h!]
	\centering
	\includegraphics[width=15cm]{./pictures/app_aspect.png}
	\caption{Orientace trojúhelníků}
\end{figure}
\clearpage


\subsection{Plošina}
\begin{figure}[h!]
	\centering
	\includegraphics[width=7cm]{./pictures/kupec_plateau_contours.png}
	\includegraphics[width=7cm]{./pictures/kupec_plateau_slope.png}
	\caption{Plošina vytvořená aplikací}
\end{figure}

\begin{figure}[h!]
	\centering
	\includegraphics[width=7cm]{./pictures/atlas_plateau.png}
	\caption{Plošina ze SW \textit{Atlas}}
\end{figure}

Plošina uprostřed území je rozeznatelná zejména z tvarů vrstevnic. Z trojúhelníkové sítě je patrné, že je trochu vyvýšená, ale i tak to chce trochu představivosti.

\subsection{Hrana}
\begin{figure}[h!]
	\centering
	\includegraphics[width=7cm]{./pictures/kupec_edge_contours.png}
	\includegraphics[width=7cm]{./pictures/kupec_edge_slope.png}
	\caption{Hrana silnice vytvořená aplikací}
\end{figure}

\begin{figure}[h!]
	\centering
	\includegraphics[width=10cm]{./pictures/atlas_edge.png}
	\caption{Hrana silnice ze SW \textit{Atlas}}
\end{figure}

Vygenerované vrstevnice v okolí silnice působí silně kostrbatým dojmem, avšak trojúhel\-ní\-ko\-vá síť poměrně věrně kopíruje tvar silnice. Jak již bylo zmíněno výše, velkým pro\-blé\-mem jsou štíhlé trojúhelníky v horní části modelu, které by neměly existovat.


\subsection{Údolnice}
\begin{figure}[h!]
	\centering
	\includegraphics[width=7.5cm]{./pictures/kupec_thalweg_contours.png}
	\includegraphics[width=7.5cm]{./pictures/kupec_thalweg_aspect.png}
	\caption{Údolnice vytvořená aplikací}
\end{figure}

\begin{figure}[h!]
	\centering
	\includegraphics[width=15cm]{./pictures/atlas_thalweg.png}
	\caption{Údolnice ze SW \textit{Atlas}}
\end{figure}

Pro názornost porovnání byla tentokrát použita orientace trojúhelníků ke světovým stranám. Z jejich natočení je patrné, že ve středu dochází ke zlomu v terénu. Vrstevnice tento úkaz hezky vystihují. Ze všech částí DMT vystihla aplikace tento tvar v porovnání s DMT z \textit{Atlasu} nejvěrněji.
\clearpage

\end{comment}

\section{Ukázka SQL dotazů}

\clearpage

\section{Závěr}


\clearpage

\section{Přílohy} 

\begin{itemize}
\item Příloha č. 1: Prezentace (prezentace.pdf)
\item Příloha č. 2: SQL dávka (davka.sql)
\end{itemize}

\clearpage

\section{Zdroje}
\begin{enumerate}
\item  \textsl{Otevřená data} [online] [cit. 28. 1. 2019].\\
Dostupné z: \href{http://www.geoportalpraha.cz/cs/opendata?fbclid=IwAR3dvAz20d2Anu-nuD9A7wC3byHUKTzDGTnlgQrmi0tC-t-SbSqN7Q5x-sA#.XE8ymWlCfIX}{http://www.geoportalpraha.cz/}

\item  \textsl{Datové sady – Národní katalog otevřených dat (NKOD)} [online] [cit. 28. 1. 2019].\\
Dostupné z: \href{https://data.gov.cz/datov\%C3\%A9-sady?dotaz=&fbclid=IwAR2jg2NvnsjO7BuHAgbXSJwh6VXFbsuXs0FONcle5ZfdPpO86Z_2C3YUQsU}{https://data.gov.cz/}

\item  \textsl{LaTeX/Source Code Listing} [online] [cit. 28. 1. 2019].\\
Dostupné z: \href{https://en.wikibooks.org/wiki/LaTeX/Source_Code_Listings}{https://en.wikibooks.org/}

\end{enumerate}

\end{document}
